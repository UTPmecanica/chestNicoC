\documentclass{article}
\usepackage[utf8]{inputenc}
\usepackage[spanish]{babel}
\usepackage{lipsum}
\usepackage{graphicx}
\usepackage{xcolor}
\usepackage{hyperref}
\usepackage{amsfonts}
\usepackage{amssymb}


\title{Pruebas con Git}
\author{Nicolas Cardona Ramirez}
\date{\today}

\begin{document}
	
	\maketitle
	
	\section{Video 2: Primeros pasos en Git}
	
	
	Para iniciar el Git se utiliza lo siguiente:
	\begin{enumerate}
		\item \textcolor{red}{git config --global user.name} seguido del usuario entre comillas
		\item \textcolor{red}{git config --global user.email} seguido del email entre comillas
		\item \textcolor{red}{git config --global color.ui true} para habilitar los colores dentro del código
	\end{enumerate}
	
	\section{Video 3: Iniciar el monitoreo}
	
	\begin{itemize}
		\item Con el comando \textcolor{blue}{git init}, para iniciar el rastreo del proyecto
		\item Con \textcolor{blue}{git status} se muestran las modificaciones del proyecto
		\item Para añadir los archivos a la sincronización se utiliza el comando \textcolor{blue}{git add}
		\item Para agregar todo al proyecto se utliza \textcolor{blue}{git add -A}
	\end{itemize}

	
	
\end{document}
