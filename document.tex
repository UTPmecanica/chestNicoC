\documentclass{article}
\usepackage[utf8]{inputenc}
\usepackage[spanish]{babel}
\usepackage{lipsum}
\usepackage{graphicx}
\usepackage{xcolor}
\usepackage{hyperref}
\usepackage{amsfonts}
\usepackage{amssymb}
\usepackage{amsmath}


\title{Pruebas con Git}
\author{Nicolas Cardona Ramirez}
\date{\today}

\begin{document}
	
	\maketitle
	
	\section{Video 2: Primeros pasos en Git}
	
	
	Para iniciar el Git se utiliza lo siguiente:
	\begin{enumerate}
		\item \textcolor{red}{git config --global user.name} seguido del usuario entre comillas
		\item \textcolor{red}{git config --global user.email} seguido del email entre comillas
		\item \textcolor{red}{git config --global color.ui true} para habilitar los colores dentro del código
	\end{enumerate}
	
	\section{Video 3: Iniciar el monitoreo}
	
	\begin{itemize}
		\item Con el comando \textcolor{blue}{git init}, para iniciar el rastreo del proyecto
		\item Con \textcolor{blue}{git status} se muestran las modificaciones del proyecto
		\item Para añadir los archivos a la sincronización se utiliza el comando \textcolor{blue}{git add}
		\item Para agregar todo al proyecto se utliza \textcolor{blue}{git add -A}
		\item Para agregar se utiliza el comando \textcolor{blue}{git commit -m} y un mensaje entre comillas para identificar la modificación
		\item Para ver una lista con los comentarios se utiliza el comando \textcolor{green}{git log}
		\item Ahora para viajar entre versiones, se crea un último commit, se utiliza el git log para copiar el código del commit y se utiliza el comando \textcolor{green}{git checkout} con el código del commit. Para ver el código en su totalidad se utiliza \textcolor{green}{git chechout master} para devolver a la versión deseada
		\item Para ``Matar los commit" se utiliza el comando \textcolor{green}{git reset} las el código del commit, el cuál se subdivide en \textcolor{blue}{git reset --soft}, el cúal no se mete con el código, esta el \textcolor{blue}{git reset --mixed} y está el comando \textcolor{blue}{git reset --hard} que borra absolutamente todo
		\item El comando \textcolor{green}{git log $>$ commits.txt} crea un documento de texto con cada uno de los commits
		\item El comando \textcolor{blue}{git help} da un pequño manual para el uso de git, si se quiere ser mas específico se utiliza git help + el nombre del comando
		
	\end{itemize}


	\section{Video 4: Ramas y Funciones}
	
	Head = Actual commit
	
	\begin{itemize}
	\item Para crear una rama se utiliza el comando \textcolor{blue}{git branch $+$ nombre de la rama}
	\item Para moverse entre ramas se utiliza el comando \textcolor{blue}{git checkout + el nombre de la rama}
	\item Para absorber una rama se utiliza el comando \textcolor{blue}{git merge} para absorber todos los commits de la rama de prueba
	\item Fast-Forward solo hace la fusion sin preguntar nada
	\item Manual Merge los cambios tienen que pasar por nosotros
	\item Para crear y cambia automaticamente a una nueva rama se utiliza el comando \textcolor{blue}{git checkout -b nueva rama}
	\end{itemize}

	\section{Video 5: GitHub}
	
	\begin{itemize}
		
	\item El comando \textcolor{green}{git clone} toma un proyecto de GitHub y lo pasa a la computadora
	\item El comando \textcolor{green}{git remote add origin} vincula el proyecto local con nuestro proyecto remoto
	\item Para comprobar el vinculo se utiliza el comando \textcolor{green}{git remote -v}
	\item Para desvincular se utiliza \textcolor{green}{git remote remove origin}
	\item Para sincronizar se utiliza \textcolor{green}{git push origin + la rama a pasar}
	
	\end{itemize}

\end{document}
